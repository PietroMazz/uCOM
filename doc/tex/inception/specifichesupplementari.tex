\subsection{Specifiche supplementari}

\subsubsection{Cronologia revisioni}


\begin{table}[htb]
	\begin{tabular}{|l|l|l|l|}
		\hline
		
		\textbf{Versione} & \textbf{Data}       & \textbf{Descrizione} & \textbf{Autore}           \\ \hline
		
		1.0      & 15/02/2019 & Prima bozza & Pietro Mazzaglia \\ \hline
		1.1      & 01/03/2019 & Revisione a fine Iterazione 4 & Pietro Mazzaglia \\ \hline
	\end{tabular}
\end{table}

\subsubsection{Introduzione}

In questo documento vengono inserite informazioni relative ai requisiti FURPS+ e ad alcuni attributi di qualità di \textit{uCOM}.

\subsubsection{Funzionalità}

\textit{Gestione degli errori}

Tutti gli errori vanno segnalati all'utilizzatore e, ove possibile, risolti via software o grazie all'interazione di un Utente.

\textit{Regole inseribili}

In alcuni casi d'uso è possibile l'integrazione o l'interazione con sistemi e servizi esterni. 

In tutti i casi d'uso potrebbero essere integrate regole di business, da rispettare per il corretto funzionamento della piattaforma nel suo contesto.

\textit{Sicurezza}

La piattaforma prevede sempre autenticazione. Le funzionalità disponibili a una categoria di utenti sono nascoste alle altre categorie.

\subsubsection{Usabilità}

\textit{Fattori umani}

La piattaforma deve essere acccessibile anche per utenti con disabilità uditive e visive.

Tutte le funzionalità devono essere disponibili con velocità e semplicità d'uso. Questi sono fattori che determinano in maniera importante la soddisfazione del cliente e dunque il successo della piattaforma.

\subsubsection{Affidabilità}

\textit{Possibilità di recupero}

Il sistema deve prevedere la possibilità di recuperare la sessione di lavoro in seguito alla chiusura anomala dell'applicazione.

Il sistema dovrebbe anche prevedere metodi di lavoro in locale, nel caso in cui non sia disponibile una connessione internet per completare le azioni richieste dall'utente.

\subsubsection{Sostenibilità}

\textit{Adattabilità}

L'applicazione deve essere sviluppata in maniera flessibile, in quanto diverse regole di business potrebbero essere applicate da Campus diversi da quello richiedente. 

Sviluppare un'applicazione flessibile permette di rivendere il software prodotto ad altre strutture, con uno sforzo di sviluppo minimo. 

\textit{Configurabilità}

Un Campus può richiedere la personalizzazione di diversi aspetti del software. Potrebbe venire richiesta la compatibilità con determinati servizi esterni.

Si potrebbe richiedere anche l'integrazione di nuove lingue o di nuovi requisiti di usabilità.

\subsubsection{Vincoli d'implementazione}

E' consigliabile lo sviluppo attraverso tecnologie Java, in quanto garantiscono portabilità, ampia disponibilità di software open source e supporto a lungo termine.

Un altro fattore determinante è la possibilità di sviluppare applicazioni mobili, riutilizzando la parte logica scritta in Java, e associandovi un'interfaccia per il sistema operativo Android (basato su Java).

\subsubsection{Componenti Open Source}

In generale, si raccomanda l'utilizzo di componenti tecnologiche Java open source per questo progetto.

Si consiglia l'utilizzo delle seguenti componenti:

- \textit{Hibernate}, framework per ORM (Object-relational mapping)

- \textit{JUnit}, framework per il testing

- \textit{MySQL}, o altro DBMS relazionale per il Database

\subsubsection{Interfacce}

\textit{Hardware}

\begin{itemize}
	\item Dispositivi mobile, touchscreen
	\item Computer desktop e laptop
\end{itemize}

\textit{Interfacce software}

Il sistema si integra con sistemi software esterni, come ad esempio:

\begin{itemize}
	\item Software per il servizio mensa
	\item Software per la gestione della biblioteca
	\item Provider e-mail per la gestione delle comunicazioni
\end{itemize}

\newpage

\subsubsection{Regole di Dominio specifiche dell'applicazione}

% Please add the following required packages to your document preamble:
% \usepackage{graphicx}
% \usepackage[normalem]{ulem}
% \useunder{\uline}{\ul}{}
\begin{table}[!htb]
	\resizebox{\textwidth}{!}{%
		\begin{tabular}{|l|l|l|l|}
			\hline
			\textbf{ID} & \textbf{Regola}                                                                                                                                                                              & \textbf{\begin{tabular}[c]{@{}l@{}}Possibilità di \\ cambiamenti\end{tabular}}                                                                       & \textbf{Origine} \\ \hline
			REGOLA1     & \begin{tabular}[c]{@{}l@{}}Regola sul giorno di prenotazione dei pasti.\\ \\ E' possibile prenotare solo pasti per il giorno successivo.\end{tabular}                                        & \begin{tabular}[c]{@{}l@{}}Media.\\ \\ E' una buona\\ norma applicata\\ in diversi Campus.\end{tabular}                                              & Servizio mensa   \\ \hline
			REGOLA2     & \begin{tabular}[c]{@{}l@{}}Regola su orario prenotazione dei pasti\\ \\ E' possibile prenotare pasti solo entro un determinato\\ orario (di solito le 19:00 / 7.00 pm)\end{tabular}          & \begin{tabular}[c]{@{}l@{}}Elevata.\\ \\ Norma applicata in\\ alcuni Campus.\\ Può variare l'ora.\end{tabular}                                       & Servizio mensa   \\ \hline
			REGOLA3     & \begin{tabular}[c]{@{}l@{}}Regola durata prestito dei libri.\\ \\ E' possibile richiedere il prestito di un libro per un \\ intervallo di tempo limitato (di solito max 6 mesi)\end{tabular} & \begin{tabular}[c]{@{}l@{}}Media.\\ \\ Quasi tutte le \\ biblioteche applicano\\ regole come questa,\\ tuttavia con tempi\\ differenti.\end{tabular} & Biblioteca       \\ \hline
		\end{tabular}%
	}
\end{table}