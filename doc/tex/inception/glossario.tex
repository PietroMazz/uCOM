\subsection{Glossario}

I seguenti termini vengono utilizzati più volte nel contesto dell'analisi e progettazione di \textit{uCOM}, pertanto se ne riportano le definizioni, onde evitare ambiguità nella lettura della documentazione.

\begin{itemize}
	
	\item \textbf{Amministratore :} membro dello staff del Campus, con funzioni di gestione dei servizi. Per vedere le operazioni a lui consentite si veda il \textit{Modello dei casi d'uso}.
	
	\item \textbf{Amministrazione : } insieme degli amministratori
	
	\item \textbf{Avviso :} strumento di comunicazione utilizzato dall'amministrazione per comunicare con gli studenti. Il suo messaggio contiene almeno \textit{Titolo} e \textit{Dettagli}.
	
	\item \textbf{Campus :} qualsiasi tipo di residenza universitaria che garantisca oltre al semplice servizio di alloggio, servizi accessori e complementari
	
	\item \textbf{Comunicazione :} strumento di comunicazione utilizzato dagli studenti verso l'amministrazione. Il suo messaggio contiene almeno un \textit{Oggetto} e un \textit{Corpo}.
	
	\item \textbf{Corso :} corso che viene organizzato all'interno del Campus. Ogni corso ha un \textit{nome} e una \textit{descrizione}.
	
	\item \textbf{Iscrizione :} iscrizione relativa a un corso del Campus. Ogni iscrizione associa uno \textit{Studente} a un \textit{Corso}.
		
	\item \textbf{Libro :} libro disponibile presso la biblioteca del Campus
	
	\item \textbf{Login :} accesso alla piattaforma
	
	\item \textbf{Pasto :} formato almeno da un primo e un secondo.
	
	\item \textbf{Prenotazione pasto :} prenotazione relativa a un pasto (pranzo o cena).
	
	\item \textbf{Registro utenti :} registro del sistema che contiene tutte le associazioni username - ruolo.
	
	\item \textbf{Registro corsi :} registro del sistema che contiene tutti i corsi che si svolgono all'interno del Campus.
	
	\item \textbf{Registro iscrizioni :} registro del sistema che contiene tutti le iscrizioni ai corsi che si svolgono all'interno del Campus.
	
	\item \textbf{Richiesta libro :} richiesta di prestito per un libro, caratterizzata dalla durata del prestito.
	
	\item \textbf{Servizio mensa :} servizio di somministrazione dei pasti con struttura interna al campus, ma la cui gestione è solitamente affidata a ditte esterne
	
	\item \textbf{Sistema/Piattaforma/Applicazione :} questi tre termini sono equivalenti nel caso in cui ci si stia riferendo alla piattaforma \textit{uCOM}. Negli altri casi sarà specificato il sistema cui si fa riferimento. 
	
	Questa classe concettuale rappresenta l'entità software che si occupa di interagire con l'Utente, elaborando i dati da lui inseriti e processando le operazioni da lui eseguite.

	\item \textbf{Studente :} persona che risiede e studia nel Campus e usufruitore dei servizi dello stesso. Per vedere le operazioni a lui consentite si veda il \textit{Modello dei casi d'uso}.
	
	\item \textbf{System Admin :} amministratore con poteri superiori, in grado di gestire gli utenti del sistema. Per vedere le operazioni a lui consentite si veda il \textit{Modello dei casi d'uso}.	
	
	\item \textbf{Utente :} qualsiasi persona che utilizzi la piattaforma, per un qualsiasi scopo. Ciascun utente ha dunque un determinato ruolo come utilizzato e deve possedere credenziali per accedere al sistema. 
	Una volta loggato, l'identificativo dell'utente è il suo username, ovvero il nome con cui effettua l'accesso alla piattaforma. 
	\textit{Studenti, Amministratori e System Admin sono Utenti}
	
\end{itemize}