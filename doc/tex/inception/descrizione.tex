\subsection{Descrizione sintetica del progetto}

Il sistema \textbf{uCOM} è una piattaforma online per la gestione automatizzata di un Campus Universitario.

Esso fornisce agli \textbf{studenti }un comodo servizio automatizzato per organizzare la propria vita all'interno del Campus, attraverso la possibilità di richiedere la fruizione di servizi (e.g. pasti alla mensa o uso della biblioteca) o l'iscrizione ai corsi organizzati internamente. 
La piattaforma permette inoltre di segnalare eventuali problemi e disservizi tramite un canale telematico di comunicazione.

E' bene sapere che alcune dei servizi del Campus possono essere gestiti da enti esterni: il servizio mensa, ad esempio, è tipicamente gestito da una ditta esterna in appalto, e la biblioteca, che seppure strutturalmente interna al campus, potrebbe avere un proprio sistema di gestione dei libri e delle prenotazioni, non integrato in \textit{uCOM}.
Altre funzionalità possono essere gestite internamente alla piattaforma, come la gestione dei corsi e il servizio di comunicazione con l'amministrazione.

L'\textbf{amministratore} rappresenta la seconda categoria di utilizzatori del sistema. Questi infatti deve poter essere in grado di postare avvisi relativi alla vita all'interno del Campus, avere una gestione dei corsi universitari e delle iscrizioni relative, e potrebbe ricoprire altre mansioni gestionali.

La gestione degli utenti e delle loro funzionalità è delegata al \textbf{System Admin}, che è in grado di amministrare le posizioni degli utilizzatori del sistema e di garantire loro accesso alla piattaforma, tramite autenticazione.