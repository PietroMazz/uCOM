\subsection{Modello dei casi d'uso}

Durante la fase di Ideazione soltanto 2 casi d'uso sono stati trattati in maniera dettagliata. Gli altri sono stati espressi in formato \textit{informale}.

\subsubsection{UC1: Prenota pasto}
% Please add the following required packages to your document preamble:
% \usepackage{longtable}
% Note: It may be necessary to compile the document several times to get a multi-page table to line up properly
\begin{longtable}{|l|l|}
	\hline
	\textbf{Nome caso d'uso} & UC1: Prenota pasto \\ \hline
	\endfirsthead
	%
	\endhead
	%
	\textbf{Portata} & Piattaforma uCOM \\ \hline
	\textbf{Livello} & Obiettivo utente \\ \hline
	\textbf{Attore primario} & Studente \\ \hline
	\textbf{\begin{tabular}[c]{@{}l@{}}Parti interessate \\ e Interessi\end{tabular}} & \begin{tabular}[c]{@{}l@{}}Studente: vuole prenotare pasto alla mensa per il giorno\\ successivo.\\ \\ Servizio Mensa: vuole ricevere la prenotazione\\ dello studente, che deve essere coerente con il menù offerto.\\ \\ Direzione Campus: vuole che i propri studenti possano\\ interagire con Servizi Esterni affiliati al Campus, come la mensa\end{tabular} \\ \hline
	\textbf{Pre-condizioni} & Lo Studente possiede un account sulla piattaforma. \\ \hline
	\textbf{Garanzia di successo} & Lo Studente ha ricevuto conferma dell'operazione. \\ \hline
	\textbf{\begin{tabular}[c]{@{}l@{}}Scenario principale \\ di successo\end{tabular}} & \begin{tabular}[c]{@{}l@{}}1. Lo Studente effettua l'accesso\\ 2. Lo Studente avvia l'operazione di prenotazione pasto.\\ 3. Lo Studente indica per quale pasto vuole prenotare\\ 4. Il Sistema mostra il menù relativo a tale pasto\\ 5. Lo Studente inserisce le proprie scelte.\\ 6. Lo Studente invia la prenotazione del pasto.\\ 7. Il Sistema elabora la prenotazione.\\ 8. Il Sistema conferma la riuscita dell'operazione.\end{tabular} \\ \hline
	\textbf{Estensioni} & \begin{tabular}[c]{@{}l@{}}*a. In qualsiasi momento.Il Sistema non è in grado di funzionare\\ correttamente in un dato momento.\\ \\  \quad   1) Il Sistema segnala l'impossibilità di eseguire l'azione.\\ \\  \quad   - Lo Studente riprova a eseguire l'azione dopo un certo periodo\\     di tempo.\end{tabular} \\ \hline
	\textbf{Estensioni} & \begin{tabular}[c]{@{}l@{}}*b. In qualsiasi momento. Il Sistema entra in uno stato di errore\\ irrisolvibile.\\ \\   \quad  1) Il Sistema termina la sessione, perdendo i dati.\\ \\     \quad- Lo Studente deve ricominciare l'operazione.\\ \\ *c. In qualsiasi momento. Lo Studente interrompe l'operazione.\\ \\ \quad 1) Il Sistema termina l'operazione.\\ \\ 3a. Lo Studente indica un'opzione non valida o possibile.\\ \\     \quad1) Il Sistema richiede nuova immissione dei dati allo Studente.\\ \\ 5a. Lo Studente inserisce informazioni non valide.\\ \\     \quad1) Il Sistema richiede nuova immissione dei dati allo Studente\\ \\ 7a. Il Sistema invia la prenotazione a un Servizio Esterno.\\ \\   \quad  1a) Il Servizio Esterno riceve correttamente la prenotazione.\\  \\         \quad\quad2) Il Sistema conferma la riuscita dell'operazione.\\ \\     \quad1b) Il Servizio Esterno rigetta la richiesta.\\ \\         \quad\quad2a) Il Sistema va in errore temporaneo.\\ \\         \quad\quad- Lo Studente può ritentare l'operazione dopo un certo\\           periodo di tempo.\\ \\         \quad\quad2b) Una regola di domio è stata violata, dunque viene\\         restituito un messaggio di errore.\\ \\          \quad\quad- Potrebbe essere richiesto l'inserimento di nuovi dati.         \\ \\ 7b. Il Sistema gestisce internamente la prenotazione.\end{tabular} \\ \hline
	\textbf{Requisiti speciali} & \begin{tabular}[c]{@{}l@{}}- Lo Studente deve inserire dati che siano conformi al menù \\ offerto dal Servizio Mensa\end{tabular} \\ \hline
	\textbf{\begin{tabular}[c]{@{}l@{}}Elenco delle varianti \\ tecnologiche e dei dati\end{tabular}} & \begin{tabular}[c]{@{}l@{}}3/5) L'inserimento delle informazioni può avvenire attraverso\\ metodi input diversi, come tastiera e mouse o un touchscreen.\\ \\ 7) La richiesta potrebbe venire rigettata dal servizio interno o \\ esterno in quanto una prenotazione è gia disponibile. In tal caso\\ si potrebbe proporre l'aggiornamento della \\prenotazione all'utente.\\ \\ 7b) L'elaborazione di sistema può avvenire internamente tramite\\ un Registro Prenotazioni relativo al giorno successivo, che viene\\ inoltrato quotidianamente al Servizio Mensa.\end{tabular} \\ \hline
	\textbf{Frequenza di ripetizione} & giornaliera \\ \hline
	\textbf{Varie} & \begin{tabular}[c]{@{}l@{}}Si potrebbe prevedere un sistema che permetta l'inserimento\\ offline e l'elaborazione non appena il servizio ritorna disponibile.\\ \\ La prenotazione viene registrata dal Sistema se viene elaborata\\ esternamente?\\ \\ Si possono prevedere meccanismi di recupero dell'istanza in\\ caso di errori gravi.\end{tabular} \\ \hline
\end{longtable}

\newpage

\subsubsection{UC2: Richiede libro}
% Please add the following required packages to your document preamble:
% \usepackage{longtable}
% Note: It may be necessary to compile the document several times to get a multi-page table to line up properly
\begin{longtable}{|l|l|}
	\hline
	\textbf{Nome caso d'uso} & UC2: Richiede libro \\ \hline
	\endfirsthead
	%
	\endhead
	%
	\textbf{Portata} & Piattaforma uCOM \\ \hline
	\textbf{Livello} & Obiettivo utente \\ \hline
	\textbf{Attore primario} & Studente \\ \hline
	\textbf{\begin{tabular}[c]{@{}l@{}}Parti interessate \\ e Interessi\end{tabular}} & \begin{tabular}[c]{@{}l@{}}Studente: vuole richiedere un libro alla biblioteca\\ del Campus.\\ \\ Biblioteca: vuole ricevere la richiesta del libro da parte\\ dello studente, in conformità con la disponibilità dei libri\\ \\ Direzione Campus: vuole che i propri studenti possano\\ interagire con Servizi affiliati al Campus, come la biblioteca\end{tabular} \\ \hline
	\textbf{Pre-condizioni} & Lo Studente possiede un account sulla piattaforma. \\ \hline
	\textbf{Garanzia di successo} & Lo Studente ha ricevuto conferma dell'operazione. \\ \hline
	\textbf{\begin{tabular}[c]{@{}l@{}}Scenario principale \\ di successo\end{tabular}} & \begin{tabular}[c]{@{}l@{}}1. Lo Studente effettua l'accesso\\ 2. Lo Studente avvia l'operazione di richiesta libro.\\ 3. Il Sistema mostra i libri disponibili\\ 4. Lo Studente inserisce il libro e la durata \\ prevista per la richiesta.\\ 5. Lo Studente invia la richiesta libro.\\ 6. Il Sistema elabora la richiesta.\\ 7. Il Sistema conferma la riuscita dell'operazione.\end{tabular} \\ \hline
	\textbf{Estensioni} & \begin{tabular}[c]{@{}l@{}}*a. In qualsiasi momento.Il Sistema non è in grado di funzionare \\ correttamente in un dato momento.\\ \\  \quad   1) Il Sistema segnala l'impossibilità di eseguire l'azione.\\ \\   \quad  - Lo Studente riprova a eseguire l'azione dopo un certo periodo\\     di tempo.\end{tabular} \\ \hline
	\textbf{Estensioni} & \begin{tabular}[c]{@{}l@{}}*b. In qualsiasi momento. Il Sistema entra in uno stato di errore\\ irrisolvibile.\\ \\   \quad  1) Il Sistema termina la sessione, perdendo i dati.\\ \\     \quad- Lo Studente deve ricominciare l'operazione.\\ \\ *c. In qualsiasi momento. Lo Studente interrompe l'operazione.\\ \\ \quad1) Il Sistema termina l'operazione.\\ \\ 4a. Lo Studente inserisce dati non validi.\\ \\     \quad1) Il Sistema richiede nuova immissione dei dati allo Studente.\\ \\ 6a. Il Sistema invia la richiesta a un Servizio Esterno.\\ \\     \quad1a) Il Servizio Esterno riceve correttamente la richiesta.\\  \\         \quad\quad2) Il Sistema conferma la riuscita dell'operazione.\\ \\     \quad1b) Il Servizio Esterno rigetta la richiesta.\\ \\         \quad\quad2a) Il Sistema va in errore temporaneo.\\ \\         \quad- Lo Studente può ritentare l'operazione dopo un certo\\           periodo di tempo.\\ \\         \quad\quad2b) Una regola di dominio è stata violata, dunque viene\\         restituito un messaggio di errore.\\ \\          \quad\quad- Potrebbe essere richiesto l'inserimento di nuovi dati.         \\ \\ 6b. Il Sistema gestisce internamente la richiesta.\end{tabular} \\ \hline
	\textbf{Requisiti speciali} & \begin{tabular}[c]{@{}l@{}}- Lo Studente deve inserire dati che siano conformi ai libri\\ disponibili in Biblioteca\end{tabular} \\ \hline
	\textbf{\begin{tabular}[c]{@{}l@{}}Elenco delle varianti \\ tecnologiche e dei dati\end{tabular}} & \begin{tabular}[c]{@{}l@{}}4) L'inserimento delle informazioni può avvenire attraverso\\ metodi input diversi, come tastiera e mouse o un touchscreen.\\ \\ 7b) L'elaborazione di sistema può avvenire internamente se il\\ servizio di gestione della biblioteca viene integrato in uCOM.\end{tabular} \\ \hline
	\textbf{Frequenza di ripetizione} & media-alta \\ \hline
	\textbf{Varie} & \begin{tabular}[c]{@{}l@{}}Si potrebbe prevedere un sistema che permetta l'inserimento\\ offline e l'elaborazione non appena il servizio ritorna disponibile.\\ \\ La richiesta viene registrata dal Sistema se viene elaborata\\ esternamente?\\ \\ Si possono prevedere meccanismi di recupero dell'istanza in\\ caso di errori gravi.\end{tabular} \\ \hline
\end{longtable}

\subsubsection{UC3: Iscrive a un corso}
% Please add the following required packages to your document preamble:
% \usepackage{longtable}
% Note: It may be necessary to compile the document several times to get a multi-page table to line up properly
\begin{longtable}{|l|l|}
	\hline
	\textbf{Nome caso d'uso} & UC3: Iscrive a un corso \\ \hline
	\endfirsthead
	%
	\endhead
	%
	\textbf{Portata} & Piattaforma uCOM \\ \hline
	\textbf{Livello} & Obiettivo utente \\ \hline
	\textbf{Attore primario} & Studente \\ \hline
	\textbf{\begin{tabular}[c]{@{}l@{}}Parti interessate \\ e Interessi\end{tabular}} & \begin{tabular}[c]{@{}l@{}}Studente: vuole iscriversi a un corso del Campus.\\ \\ Direzione Campus: vuole che i propri studenti possano\\ iscriversi ai corsi tramite la piattaforma fornita\end{tabular} \\ \hline
	\textbf{Pre-condizioni} & Lo Studente possiede un account sulla piattaforma. \\ \hline
	\textbf{Garanzia di successo} & Lo Studente ha ricevuto conferma dell'operazione. \\ \hline
	\textbf{\begin{tabular}[c]{@{}l@{}}Scenario principale \\ di successo\end{tabular}} & \begin{tabular}[c]{@{}l@{}}1. Lo Studente effettua l'accesso\\ 2. Lo Studente avvia l'operazione di iscrizione a un corso.\\ 3. Lo Studente indica il corso cui vuole iscriversi.\\ 4. Lo Studente invia l'iscrizione al corso.\\ 5. Il Sistema elabora l'iscrizione.\\ 6. Il Sistema conferma la riuscita dell'operazione.\end{tabular} \\ \hline
	\textbf{Estensioni} & \begin{tabular}[c]{@{}l@{}}*a. In qualsiasi momento.Il Sistema non è in grado di funzionare \\ correttamente in un dato momento.\\ \\     1) Il Sistema segnala l'impossibilità di eseguire l'azione.\\ \\     - Lo Studente riprova a eseguire l'azione dopo un certo periodo\\     di tempo.\\ \\ *b. In qualsiasi momento. Il Sistema entra in uno stato di errore\\ irrisolvibile.\\ \\     1) Il Sistema termina la sessione, perdendo i dati.\\ \\     - Lo Studente deve ricominciare l'operazione.\\ \\ *c. In qualsiasi momento. Lo Studente interrompe l'operazione.\\ \\ 1) Il Sistema termina l'operazione.\\ \\ 3a. Lo Studente indica un'opzione non valida o disponibile.\\ \\     1) Il Sistema richiede nuova immissione dei dati allo Studente.\\ \\ 5a. Lo studente non può iscriversi al corso selezionato.\\ \\   1a) Il Sistema indica il motivo per cui non è possibile iscriversi\\ e termina l'operazione.\end{tabular} \\ \hline
	\textbf{Requisiti speciali} & \begin{tabular}[c]{@{}l@{}}- Lo Studente deve inserire dati che siano conformi ai corsi\\ offerti dal Campus\end{tabular} \\ \hline
	\textbf{\begin{tabular}[c]{@{}l@{}}Elenco delle varianti \\ tecnologiche e dei dati\end{tabular}} & \begin{tabular}[c]{@{}l@{}}3) L'inserimento delle informazioni può avvenire attraverso\\ metodi input diversi, come tastiera e mouse o un touchscreen.\\ \\ 3) I Corsi disponibili sono quelli del Registro Corsi di uCOM.\end{tabular} \\ \hline
	\textbf{Frequenza di ripetizione} & bassa \\ \hline
	\textbf{Varie} & \begin{tabular}[c]{@{}l@{}}Si potrebbe prevedere un sistema che permetta l'inserimento\\ offline e l'elaborazione non appena il servizio ritorna disponibile.\\ \\ Si possono prevedere meccanismi di recupero dell'istanza in\\ caso di errori gravi.\end{tabular} \\ \hline
\end{longtable}

\newpage

\subsubsection{UC4: Invia comunicazione}

	% Please add the following required packages to your document preamble:
	% \usepackage{longtable}
	% Note: It may be necessary to compile the document several times to get a multi-page table to line up properly
	\begin{longtable}[c]{|l|l|}
		\hline
		\textbf{Nome caso d'uso}                                                                          & UC4: Invia comunicazione                                                                           \\ \hline
		\endfirsthead
		%
		\endhead
		%
		\textbf{Portata}                                                                                  & Piattaforma uCOM                                                                                                                                                                                                                                                                                                                                                                                                                                                                                                                                                                                                                                                                                                                                                                                                                                                                                                                                                                                                                                                                                                                       \\ \hline
		\textbf{Livello}                                                                                  & Obiettivo utente                                                                                                                                                                                                                                                                                                                                                                                                                                                                                                                                                                                                                                                                                                                                                                                                                                                                                                                                                                                                                                                                                                                       \\ \hline
		\textbf{Attore primario}                                                                          & Studente                                                                                                                                                                                                                                                                                                                                                                                                                                                                                                                                                                                                                                                                                                                                                                                                                                                                                                                                                                                                                                                                                                                               \\ \hline
		\textbf{\begin{tabular}[c]{@{}l@{}}Parti interessate \\ e Interessi\end{tabular}}                 & \begin{tabular}[c]{@{}l@{}}\textit{Studente: }vuole inviare una comunicazione relativa\\ alla vita all'interno del Campus\\ \\ \textit{Amministrazione: }vuole potere ricevere la comunicazione\\  dello studente\\ \\ \textit{Direzione Campus:} vuole che la comunicazione \\ avvenga in maniera rapida, sicura e affidabile\end{tabular}                                                                                                                                                                                                                                                                                                                                                                                                                                                                                                                                                                                                                                                                                                                                                                                                                       \\ \hline
		\textbf{Pre-condizioni}                                                                           & Lo Studente possiede un account sulla piattaforma.                                                                                                                                                                                                                                                                                                                                                                                                                                                                                                                                                                                                                                                                                                                                                                                                                                                                                                                                                                                                                                                                                     \\ \hline
		\textbf{Garanzia di successo}                                                                     & Lo Studente ha ricevuto conferma dell'operazione.                                                                                                                                                                                                                                                                                                                                                                                                                                                                                                                                                                                                                                                                                                                                                                                                                                                                                                                                                                                                                                                                                      \\ \hline
		\textbf{\begin{tabular}[c]{@{}l@{}}Scenario principale \\ di successo\end{tabular}}               & \begin{tabular}[c]{@{}l@{}}1. Lo Studente effettua l'accesso\\ 2. Lo Studente avvia l'operazione di invio della comunicazione.\\ 3. Lo Studente inserisce oggetto e corpo della comunicazione.\\ 4. Lo Studente invia la comunicazione.\\ 5. Il Sistema elabora la comunicazione.\\ 6. Il Sistema conferma la riuscita dell'operazione.\end{tabular}                                                                                                                                                                                                                                                                                                                                                                                                                                                                                                                                                                                                                                                                                                                                                                                                 \\ \hline
		\textbf{Estensioni}                                                                               & \begin{tabular}[c]{@{}l@{}}*a. In qualsiasi momento.Il Sistema non è in grado di funzionare \\ correttamente in un dato momento.\\ \\  \quad   1) Il Sistema segnala l'impossibilità di eseguire l'azione.\\ \\  \quad   - Lo Studente riprova a eseguire l'azione dopo un certo periodo\\     di tempo.\\ \\  *b. In qualsiasi momento. Il Sistema entra in uno stato di errore\\ irrisolvibile.\\ \\  \quad   1) Il Sistema termina la sessione, perdendo i dati.\\ \\  \quad   - Lo Studente deve ricominciare l'operazione.\\ \\ *c. In qualsiasi momento. Lo Studente interrompe l'operazione. \\ \\ \quad 1)Il Sistema termina l'operazione. \\ \\ 3a. Lo Studente inserisce informazioni non valide.\\ \\  \quad   1) Il Sistema richiede nuova immissione dei dati allo Studente.\\ \\ 5a. Il Sistema invia il messaggio a un Servizio Esterno.\\ \\  \quad   1a) Il Servizio Esterno riceve correttamente il messaggio.\\  \\    \quad \quad     2) Il Sistema conferma la riuscita dell'operazione.\\ \\  \quad   1b) Il Servizio Esterno rigetta la richiesta.\\ \\    \quad \quad     - Il Sistema va in errore temporaneo.\\ \\   \quad \quad      - Lo Studente può ritentare l'operazione dopo un certo\\           periodo di tempo.\\ \\ 5b. Il Sistema gestisce internamente il messaggio.\end{tabular} \\ \hline
		\textbf{Requisiti speciali}                                                                       & \begin{tabular}[c]{@{}l@{}}- Lo Studente deve poter inserire le informazioni nella propria\\ lingua o nella lingua di comunicazione del Campus.\end{tabular}                                                                                                                                                                                                                                                                                                                                                                                                                                                                                                                                                                                                                                                                                                                                                                                                                                                                                                                                                                           \\ \hline
		\textbf{\begin{tabular}[c]{@{}l@{}}Elenco delle varianti \\ tecnologiche e dei dati\end{tabular}} & \begin{tabular}[c]{@{}l@{}}3) L'inserimento delle informazioni può avvenire attraverso\\ metodi d'input diversi, come tastiera e mouse o un touchscreen.\\ \\ 5) L'elaborazione di sistema può avvenire internamente o\\ esternamente alla piattaforma uCOM.\end{tabular}
		\\ \hline
		\textbf{Frequenza di ripetizione}                                                                 & quasi giornaliera                                                                                                                                                                                                                                                                                                                                                                                                                                                                                                                                                                                                                                                                                                                                                                                                                                                                                                                                                                                                                                                                                                                      \\ \hline
		\textbf{Varie}                                                                                    & \begin{tabular}[c]{@{}l@{}}Si potrebbe prevedere un sistema che permetta l'inserimento\\ offline e l'elaborazione non appena il servizio ritorna disponibile.\\ \\ Si potrebbe integrare il servizio esterno all'interno della\\ piattaforma, piuttosto che inviare esternamente il messaggio \\ per l'elaborazione.\\ \\ Il messaggio viene memorizzato dal Sistema se viene elaborato\\ esternamente?\\ \\ Si possono prevedere meccanismi di recupero dell'istanza in\\ caso di errori gravi. \\ \\ Si potrebbe prevedere l'aggiunta di allegati alla comunicazione.\end{tabular}                                                                                                                                                                                                                                                                                                                                                                                                                                                                                                                                                                                                                 \\ \hline
	\end{longtable}

\subsubsection{UC5: Gestisce corso}
% Please add the following required packages to your document preamble:
% \usepackage[normalem]{ulem}
% \useunder{\uline}{\ul}{}
% \usepackage{longtable}
% Note: It may be necessary to compile the document several times to get a multi-page table to line up properly
\begin{longtable}{|l|l|}
	\hline
	\textbf{Nome caso d'uso} & UC5: Gestisce corso \\ \hline
	\endfirsthead
	%
	\endhead
	%
	\textbf{Portata} & Piattaforma uCOM \\ \hline
	\textbf{Livello} & Obiettivo utente (CRUD) \\ \hline
	\textbf{Attore primario} & Amministratore \\ \hline
	\textbf{\begin{tabular}[c]{@{}l@{}}Parti interessate \\ e Interessi\end{tabular}} & \begin{tabular}[c]{@{}l@{}}Amministratore: vuole poter gestire la creazione di un corso\\ svolto internamente al Campus\\ \\ Direzione Campus: vuole che la gestione dei corsi che si\\ svolgono all'interno del Campus siano gestite internamente dal\\ personale del Campus (Amministrazione)\end{tabular} \\ \hline
	\textbf{Pre-condizioni} & L'Amministratore possiede un account sulla piattaforma \\ \hline
	\textbf{Garanzia di successo} & \begin{tabular}[c]{@{}l@{}}Un nuovo corso è stato creato.\\ L'Amministratore ha ricevuto conferma dell'operazione.\end{tabular} \\ \hline
	\textbf{\begin{tabular}[c]{@{}l@{}}Scenario principale \\ di successo\end{tabular}} & \begin{tabular}[c]{@{}l@{}}1. L'Amministratore effettua l'accesso.\\ 2. L'Amministratore avvia la creazione di un corso.\\ 3. L'Amministratore inserisce nome e descrizione del corso.\\ 4. L'Amministratore aggiunge il corso al Sistema.\\ 5. Il Sistema aggiunge il corso al proprio Registro Corsi.\\ 6. Il Sistema conferma la riuscita dell'operazione.\end{tabular} \\ \hline
	\textbf{Estensioni} & \begin{tabular}[c]{@{}l@{}}*a. In qualsiasi momento.Il Sistema non è in grado di funzionare\\ correttamente in un dato momento.\\ \\ \quad1) Il Sistema segnala l’impossibilità a di eseguire l’azione. \\ \\ \quad- L'Amministratore riprova a eseguire l’azione dopo un certo \\ periodo di tempo.\\ \\ *b. In qualsiasi momento. Il Sistema entra in uno stato di errore \\ irrisolvibile. \\ \\ \quad1) Il Sistema termina la sessione, perdendo i dati. \\ \\ \quad- L'Amministratore deve ricominciare l’operazione.\\ \\ *c. In qualsiasi momento. L'Amministratore interrompe \\ l’operazione. \\ \\ \quad1) Il Sistema termina l’operazione.\\ \\ 2a. L'Amministratore avvia la lettura dei dati di un corso.\\ \\ \quad1) L'Amministratore inserisce il nome del Corso.\\ \\ \quad2) L'Amministratore richiede la lettura dei dati del corso.\\ \\ \quad3) Il Sistema preleva i dati del corso richiesto \\ dal Registro Corsi.\\ \\ \quad4) Il Sistema restituisce i dati del corso richiesto.\end{tabular} \\ \hline
	\textbf{Estensioni} & \begin{tabular}[c]{@{}l@{}}2b. L'Amministratore avvia la modifica di un corso.\\ \\ \quad1) L'Amministratore inserisce le informazioni del corso\\ da modificare.\\ \\ \quad2) L'Amministratore modifica il corso.\\ \\ \quad3) Il Sistema aggiorna il corso sul Registro Corsi.\\ \\ \quad4) Il Sistema conferma la riuscita dell'operazione.\\ \\ 2c. L'Amministratore avvia l'eliminazione di un corso.\\ \\ \quad1) L'Amministratore inserisce le informazioni del corso\\ da eliminare.\\ \\ \quad2) L'Amministratore elimina il corso.\\ \\ \quad3) Il Sistema elimina il corso dal Registro Corsi.\\ \\ \quad4) Il Sistema conferma la riuscita dell'operazione.\\ \\ 3a. L'Amministratore inserisce informazioni non valide. \\ \\ \quad1) Il Sistema richiede nuova immissione dei dati \\ all'Amministratore\end{tabular} \\ \hline
	\textbf{Requisiti speciali} & Nessuno \\ \hline
	\textbf{\begin{tabular}[c]{@{}l@{}}Elenco delle varianti \\ tecnologiche e dei dati\end{tabular}} & \begin{tabular}[c]{@{}l@{}}3) L'inserimento delle informazioni può avvenire attraverso\\ metodi input diversi, come tastiera e mouse o un touchscreen.\end{tabular} \\ \hline
	\textbf{Frequenza di ripetizione} & frequenza medio-bassa \\ \hline
	\textbf{Varie} & \begin{tabular}[c]{@{}l@{}}Si potrebbero aggiungere nuove informazioni da memorizzare sul\\ Registro Corsi.\end{tabular} \\ \hline
\end{longtable}


\newpage

\subsubsection{UC6: Gestisce iscrizione corso}
% Please add the following required packages to your document preamble:
% \usepackage[normalem]{ulem}
% \useunder{\uline}{\ul}{}
% \usepackage{longtable}
% Note: It may be necessary to compile the document several times to get a multi-page table to line up properly
\begin{longtable}{|l|l|}
	\hline
	\textbf{Nome caso d'uso} & UC6: Gestisce iscrizione corso \\ \hline
	\endfirsthead
	%
	\endhead
	%
	\textbf{Portata} & Piattaforma uCOM \\ \hline
	\textbf{Livello} & Obiettivo utente (CRUD) \\ \hline
	\textbf{Attore primario} & Amministratore \\ \hline
	\textbf{\begin{tabular}[c]{@{}l@{}}Parti interessate \\ e Interessi\end{tabular}} & \begin{tabular}[c]{@{}l@{}}Amministratore: vuole poter gestire la creazione di\\ un'iscrizione a un corso svolto internamente al Campus\\ \\ Direzione Campus: vuole che i corsi che si\\ svolgono all'interno del Campus siano gestite internamente dal\\ personale del Campus (Amministrazione)\end{tabular} \\ \hline
	\textbf{Pre-condizioni} & L'Amministratore possiede un account sulla piattaforma \\ \hline
	\textbf{Garanzia di successo} & \begin{tabular}[c]{@{}l@{}}Una nuova iscrizione è stata creata.\\ L'Amministratore ha ricevuto conferma dell'operazione.\end{tabular} \\ \hline
	\textbf{\begin{tabular}[c]{@{}l@{}}Scenario principale \\ di successo\end{tabular}} & \begin{tabular}[c]{@{}l@{}}1. L'Amministratore effettua l'accesso.\\ 2. L'Amministratore avvia l'iscrizione a un corso.\\ 3. L'Amministratore inserisce nome del corso e \\identificativo dello studente.\\ 4. L'Amministratore aggiunge l'iscrizione al Sistema.\\ 5. Il Sistema aggiunge l'iscrizione al proprio Registro Iscrizioni.\\ 6. Il Sistema conferma la riuscita dell'operazione.\end{tabular} \\ \hline
	\textbf{Estensioni} & \begin{tabular}[c]{@{}l@{}}*a. In qualsiasi momento.Il Sistema non è in grado di funzionare\\ correttamente in un dato momento.\\ \\ \quad1) Il Sistema segnala l’impossibilità a di eseguire l’azione. \\ \\ \quad- L'Amministratore riprova a eseguire l’azione dopo un certo \\ periodo di tempo.\\ \\ *b. In qualsiasi momento. Il Sistema entra in uno stato di errore \\ irrisolvibile. \\ \\ \quad1) Il Sistema termina la sessione, perdendo i dati. \\ \\ \quad- L'Amministratore deve ricominciare l’operazione.\\ \\ *c. In qualsiasi momento. L'Amministratore interrompe \\ l’operazione. \\ \\ \quad1) Il Sistema termina l’operazione.\\ \\ 2a. L'Amministratore avvia la lettura degli iscritti a un corso.\\ \\ \quad1) L'Amministratore inserisce il nome del Corso.\\ \\ \quad2) L'Amministratore richiede la lettura degli iscritti al corso.\\ \\ \quad3) Il Sistema preleva i dati relativi al corso richiesto \\ dal Registro Iscrizioni.\\ \\ \quad4) Il Sistema restituisce gli iscritti al corso richiesto.\end{tabular} \\ \hline
	\textbf{Estensioni} & \begin{tabular}[c]{@{}l@{}}2b. L'Amministratore avvia la modifica di un'iscrizione.\\ \\ \quad1) L'Amministratore inserisce le informazioni\\ relative a un'iscrizione da modificare.\\ \\ \quad2) L'Amministratore modifica l'iscrizione.\\ \\ \quad3) Il Sistema aggiorna l'iscrizione sul Registro Iscrizioni.\\ \\ \quad4) Il Sistema conferma la riuscita dell'operazione.\\ \\ 2c. L'Amministratore avvia l'eliminazione di un'iscrizione.\\ \\ \quad1) L'Amministratore inserisce le informazioni dell'iscrizione\\ da eliminare.\\ \\ \quad2) L'Amministratore elimina l'iscrizione.\\ \\ \quad3) Il Sistema elimina l'iscrizione dal Registro Iscrizioni.\\ \\ \quad4) Il Sistema conferma la riuscita dell'operazione.\\ \\ 3a. L'Amministratore inserisce informazioni non valide. \\ \\ \quad1) Il Sistema richiede nuova immissione dei dati \\ all'Amministratore\end{tabular} \\ \hline
	\textbf{Requisiti speciali} & Nessuno \\ \hline
	\textbf{\begin{tabular}[c]{@{}l@{}}Elenco delle varianti \\ tecnologiche e dei dati\end{tabular}} & \begin{tabular}[c]{@{}l@{}}3) L'inserimento delle informazioni può avvenire attraverso\\ metodi input diversi, come tastiera e mouse o un touchscreen.\\ \\ Le iscrizioni devono essere relative a\\ corsi esistenti nel Registro Corsi.\end{tabular} \\ \hline
	\textbf{Frequenza di ripetizione} & frequenza bassa \\ \hline
	\textbf{Varie} & \begin{tabular}[c]{@{}l@{}}Si potrebbero aggiungere nuove informazioni da memorizzare sul\\ Registro Iscrizioni.\end{tabular} \\ \hline
\end{longtable}

\subsubsection{UC7: Invia avviso}

\begin{longtable}{|l|l|}
	\hline
	\textbf{Nome caso d'uso}                                                                          & UC7: Invia avviso                                                                                                                                                                                                                                                                                                                                                                                                                                                                                                                                                                                                                                                                                                                                                                                                                                                                                                                                                                                                                                                                                                                                            \\ \hline
	\endfirsthead
	%
	\endhead
	%
	\textbf{Portata}                                                                                  & Piattaforma uCOM                                                                                                                                                                                                                                                                                                                                                                                                                                                                                                                                                                                                                                                                                                                                                                                                                                                                                                                                                                                                                                                                                                                                             \\ \hline
	\textbf{Livello}                                                                                  & Obiettivo utente                                                                                                                                                                                                                                                                                                                                                                                                                                                                                                                                                                                                                                                                                                                                                                                                                                                                                                                                                                                                                                                                                                                                             \\ \hline
	\textbf{Attore primario}                                                                          & Amministratore                                                                                                                                                                                                                                                                                                                                                                                                                                                                                                                                                                                                                                                                                                                                                                                                                                                                                                                                                                                                                                                                                                                                               \\ \hline
	\textbf{\begin{tabular}[c]{@{}l@{}}Parti interessate \\ e Interessi\end{tabular}}                 & \begin{tabular}[c]{@{}l@{}}Amministratore: vuole inviare un avviso relativo\\ alla vita all'interno del Campus\\ \\ Studente: vuole potere ricevere la comunicazione\\  dall'amministrazione\\ \\ Direzione Campus: vuole che la comunicazione \\ avvenga in maniera rapida, sicura e affidabile\end{tabular}                                                                                                                                                                                                                                                                                                                                                                                                                                                                                                                                                                                                                                                                                                                                                                                                                                                \\ \hline
	\textbf{Pre-condizioni}                                                                           & L'Amministratore possiede un account sulla piattaforma.                                                                                                                                                                                                                                                                                                                                                                                                                                                                                                                                                                                                                                                                                                                                                                                                                                                                                                                                                                                                                                                                                                      \\ \hline
	\textbf{Garanzia di successo}                                                                     & L'Amministratore ha ricevuto conferma dell'operazione.                                                                                                                                                                                                                                                                                                                                                                                                                                                                                                                                                                                                                                                                                                                                                                                                                                                                                                                                                                                                                                                                                                       \\ \hline
	\textbf{\begin{tabular}[c]{@{}l@{}}Scenario principale \\ di successo\end{tabular}}               & \begin{tabular}[c]{@{}l@{}}1. L'Amministratore effettua l'accesso\\ 2. L'Amministratore avvia l'operazione di invio dell'avviso.\\ 3. L'Amministratore inserisce titolo e dettagli dell'avviso.\\ 4. L'Amministratore invia l'avviso.\\ 5. Il Sistema elabora l'avviso.\\ 6. Il Sistema conferma la riuscita dell'operazione.\end{tabular}                                                                                                                                                                                                                                                                                                                                                                                                                                                                                                                                                                                                                                                                                                                                                                                                                      \\ \hline
	\textbf{Estensioni}                                                                               & \begin{tabular}[c]{@{}l@{}}*a. In qualsiasi momento.Il Sistema non è in grado di funzionare \\ correttamente in un dato momento.\\ \\ \quad 1) Il Sistema segnala l'impossibilità di eseguire l'azione.\\ \\   \quad  - L'Amministratore riprova a eseguire l'azione dopo un certo \\ periodo di tempo.\\ \\ *b. In qualsiasi momento. Il Sistema entra in uno stato di errore\\ irrisolvibile.\\ \\  \quad   1) Il Sistema termina la sessione, perdendo i dati.\\ \\ \quad    - L'Amministratore deve ricominciare l'operazione.\\ \\ *c. In qualsiasi momento. L'Amministratore \\interrompe l'operazione. \\ \\ \quad 1)Il Sistema termina l'operazione. \\ \\ 3a. L'Amministratore inserisce informazioni non valide.\\ \\  \quad   1) Il Sistema richiede nuova immissione dei dati \\ all'Amministratore.\\ \\ 5a. Il Sistema invia il messaggio a un Servizio Esterno.\\ \\  \quad   1a) Il Servizio Esterno riceve correttamente il messaggio.\\  \\   \quad \quad      2) Il Sistema conferma la riuscita dell'operazione.\\ \\  \quad   1b) Il Servizio Esterno rigetta la richiesta.\\ \\   \quad \quad      - Il Sistema va in errore temporaneo.\\ \\    \quad \quad     - L'Amministratore può ritentare l'operazione dopo un certo\\           periodo di tempo.\\ \\ 5b. Il Sistema gestisce internamente il messaggio.\end{tabular} \\ \hline
	\textbf{Requisiti speciali}                                                                       & \begin{tabular}[c]{@{}l@{}}- L'Amministratore deve poter inserire le informazioni nella \\ lingua di comunicazione del Campus.\end{tabular}                                                                                                                                                                                                                                                                                                                                                                                                                                                                                                                                                                                                                                                                                                                                                                                                                                                                                                                                                                                                                  \\ \hline
	\textbf{\begin{tabular}[c]{@{}l@{}}Elenco delle varianti \\ tecnologiche e dei dati\end{tabular}} & \begin{tabular}[c]{@{}l@{}}3) L'inserimento delle informazioni può avvenire attraverso\\ metodi d'input diversi, come tastiera e mouse o un touchscreen.\\ \\ 5) L'elaborazione di sistema può avvenire internamente o\\ esternamente alla piattaforma uCOM.\end{tabular}                                                                                                                                                                                                                                                                                                                                                                                                                                                                                                                                                                                                                                                                                                                                                                                                                                                                                            \\ \hline
	\textbf{Frequenza di ripetizione}                                                                 & quasi giornaliera                                                                                                                                                                                                                                                                                                                                                                                                                                                                                                                                                                                                                                                                                                                                                                                                                                                                                                                                                                                                                                                                                                                                            \\ \hline
	\textbf{Varie}                                                                                    & \begin{tabular}[c]{@{}l@{}}Si potrebbe prevedere un sistema che permetta l'inserimento\\ offline e l'elaborazione non appena il servizio ritorna disponibile.\\ \\ Si potrebbe integrare il servizio esterno all'interno della\\ piattaforma, piuttosto che inviare esternamente l'avviso  \\ per l'elaborazione.\\ \\ L'avviso viene memorizzato dal Sistema se viene elaborato\\ esternamente?\\ \\ Si possono prevedere meccanismi di recupero dell'istanza in\\ caso di errori gravi.\end{tabular}                                                                                                                                                                                                                                                                                                                                                                                                                                                                                                                                                                                                                                               \\ \hline
\end{longtable}


\subsubsection{UC8: Gestisce utente}
% Please add the following required packages to your document preamble:
% \usepackage[normalem]{ulem}
% \useunder{\uline}{\ul}{}
% \usepackage{longtable}
% Note: It may be necessary to compile the document several times to get a multi-page table to line up properly
\begin{longtable}{|l|l|}
	\hline
	\textbf{Nome caso d'uso} & UC8: Gestisce utente \\ \hline
	\endfirsthead
	%
	\endhead
	%
	\textbf{Portata} & Piattaforma uCOM \\ \hline
	\textbf{Livello} & Obiettivo utente (CRUD) \\ \hline
	\textbf{Attore primario} & System Admin \\ \hline
	\textbf{\begin{tabular}[c]{@{}l@{}}Parti interessate \\ e Interessi\end{tabular}} & \begin{tabular}[c]{@{}l@{}}SystemAdmin: vuole poter gestire la creazione di un utente\\ per poter accedere alla piattaforma\\ \\ Direzione Campus: necessita che l'accesso di ogni utente\\ sia verificato, per garantire sicurezza al sistema, e che solo un\\ utente autorizzato, quale il SystemAdmin possa modificare\\ gli utenti che possono accedere al Sistema\end{tabular} \\ \hline
	\textbf{Pre-condizioni} & Il SystemAdmin possiede un account sulla piattaforma \\ \hline
	\textbf{Garanzia di successo} & Un nuovo utente può avere accesso alla piattaforma \\ \hline
	\textbf{\begin{tabular}[c]{@{}l@{}}Scenario principale \\ di successo\end{tabular}} & \begin{tabular}[c]{@{}l@{}}1. Il SystemAdmin effettua l'accesso.\\ 2. Il SystemAdmin avvia la creazione di un utente.\\ 3. Il SystemAdmin inserisce username e ruolo per l'utente.\\ 4. Il SystemAdmin aggiunge l'utente al Sistema.\\ 5. Il Sistema aggiunge l'utente al proprio Registro Utenti.\\ 6. Il Sistema conferma la riuscita dell'operazione.\end{tabular} \\ \hline
	\textbf{Estensioni} & \begin{tabular}[c]{@{}l@{}}*a. In qualsiasi momento.Il Sistema non `e in grado di funzionare\\ correttamente in un dato momento.\\ \\ \quad1) Il Sistema segnala l’impossibilità a di eseguire l’azione. \\ \\ \quad- Il SystemAdmin riprova a eseguire l’azione dopo un certo \\ periodo di tempo.\\ \\ *b. In qualsiasi momento. Il Sistema entra in uno stato di errore \\ irrisolvibile. \\ \\ \quad1) Il Sistema termina la sessione, perdendo i dati. \\ \\ \quad- Il SystemAdmin deve ricominciare l’operazione.\\ \\ *c. In qualsiasi momento. Il SystemAdmin \\ interrompe l’operazione. \\ \\ \quad1) Il Sistema termina l’operazione.\\ \\ 2a. Il SystemAdmin avvia la lettura dei dati di un utente.\\ \\ \quad 1) Il SystemAdmin inserisce l'identificativo dell'utente.\\ \\ \quad 2) Il SystemAdmin richiede la lettura dell'utente.\\ \\ \quad 3) Il Sistema preleva i dati dell'utente richiesto \\ dal Registro Utenti.\\ \\ \quad4) Il Sistema restituisce i dati dell'utente richiesto.\end{tabular} \\ \hline
	\textbf{Estensioni} & \begin{tabular}[c]{@{}l@{}}2b. Il SystemAdmin avvia la modifica di un utente.\\ \\ \quad1) Il SystemAdmin inserisce le informazioni dell'utente\\da modificare.\\ \\ \quad2) Il SystemAdmin modifica l'utente.\\ \\ \quad3) Il Sistema aggiorna l'utente sul RegistroUtenti.\\ \\ \quad4) Il Sistema conferma la riuscita dell'operazione.\\ \\ 2c. Il SystemAdmin avvia l'eliminazione di un utente.\\ \\ \quad1) Il SystemAdmin inserisce le informazioni dell'utente\\  da eliminare.\\ \\ \quad2) Il SystemAdmin elimina l'utente.\\ \\ \quad 3) Il Sistema elimina l'utente dal RegistroUtenti.\\ \\ \quad4) Il Sistema conferma la riuscita dell'operazione.\\ \\ 3a. Il SystemAdmin inserisce informazioni non valide. \\ \\ \quad1) Il Sistema richiede nuova immissione dei dati\\ al SystemAdmin\end{tabular} \\ \hline
	\textbf{Requisiti speciali} & Nessuno \\ \hline
	\textbf{\begin{tabular}[c]{@{}l@{}}Elenco delle varianti \\ tecnologiche e dei dati\end{tabular}} & \begin{tabular}[c]{@{}l@{}}3) L'inserimento delle informazioni può avvenire attraverso\\ metodi d'input diversi, come tastiera e mouse o un touchscreen.\end{tabular} \\ \hline
	\textbf{Frequenza di ripetizione} & frequenza medio-bassa \\ \hline
	\textbf{Varie} & \begin{tabular}[c]{@{}l@{}}Si potrebbero aggiungere nuove informazioni da memorizzare sul\\ Registro Utenti.\end{tabular} \\ \hline
\end{longtable}

\subsubsection{UC9: Effettua accesso}
\begin{longtable}[c]{|l|l|}
		\hline
		\textbf{Nome caso d'uso}                                                                          & UC9: Effettua accesso                                                                                                                                                                                                                                                                                                                                                                                                      \\ \hline
		\endfirsthead
		%
		\endhead
		%
		\textbf{Portata}                                                                                  & Piattaforma uCOM                                                                                                                                                                                                                                                                                                                                                                                                           \\ \hline
		\textbf{Livello}                                                                                  & Sottofunzione                                                                                                                                                                                                                                                                                                                                                                                                              \\ \hline
		\textbf{Attore primario}                                                                          & Utente (Studente/Amministratore/System Admin)                                                                                                                                                                                                                                                                                                                                                                              \\ \hline
		\textbf{\begin{tabular}[c]{@{}l@{}}Parti interessate \\ e Interessi\end{tabular}}                 & \begin{tabular}[c]{@{}l@{}}Utente: vuole accedere alle funzionalità a lui riservate\\ \\ Direzione Campus: necessita che l'accesso di ogni utente\\ sia verificato, per garantire sicurezza al sistema\end{tabular}                                                                                                                                                                                                        \\ \hline
		\textbf{Pre-condizioni}                                                                           & Nessuna                                                                                                                                                                                                                                                                                                                                                                                                                    \\ \hline
		\textbf{Garanzia di successo}                                                                     & L'Utente ha accesso alle funzionalità del Sistema                                                                                                                                                                                                                                                                                                                                                                          \\ \hline
		\textbf{\begin{tabular}[c]{@{}l@{}}Scenario principale \\ di successo\end{tabular}}               & \begin{tabular}[c]{@{}l@{}}1. L'Utente avvia il Sistema.\\ 2. Il Sistema richiede le credenziali\\ 3. L'Utente inserisce le credenziali\\ 4. Il Sistema verifica le credenziali.\\ 5. Il Sistema garantisce all'Utente l'accesso alle funzionalità \\ a lui riservate.\end{tabular}                                                                                                                                        \\ \hline
		\textbf{Estensioni}                                                                               & \begin{tabular}[c]{@{}l@{}}*a. In qualsiasi momento.Il Sistema non è in grado di funzionare \\ correttamente in un dato momento.\\ \\   \quad  1) Il Sistema segnala l'impossibilità di eseguire l'azione.\\ \\   \quad  - L'Utente riprova a eseguire l'azione dopo un certo periodo\\     di tempo.\\ \\ 3a. L'Utente inserisce credenziali non valide.\\ \\  \quad   1) Il Sistema richiede nuove credenziali all'Utente.\end{tabular} \\ \hline
		\textbf{Requisiti speciali}                                                                       & Nessuno                                                                                                                                                                                                                                                                                                                                                                                                                    \\ \hline
		\textbf{\begin{tabular}[c]{@{}l@{}}Elenco delle varianti \\ tecnologiche e dei dati\end{tabular}} & \begin{tabular}[c]{@{}l@{}}3) L'inserimento delle informazioni può avvenire attraverso\\ metodi input diversi, come tastiera e mouse o un touchscreen.\\ \\ 4) La verifica delle credenziali può necessitare di \\ una connessione a Internet.\end{tabular}                                                                                                                                                                \\ \hline
		\textbf{Frequenza di ripetizione}                                                                 & alta frequenza giornaliera                                                                                                                                                                                                                                                                                                                                                                                                 \\ \hline
		\textbf{Varie}                                                                                    & \begin{tabular}[c]{@{}l@{}}Si potrebbe integrare un servizio esterno per l'accesso, \\ ad esempio tramite account Google+ o Facebook. \\ In tal caso la verifica non dipende dal sistema uCOM.\end{tabular}                                                                                                                                                                                                                \\ \hline
	\end{longtable}
