% Please add the following required packages to your document preamble:
% \usepackage{longtable}
% Note: It may be necessary to compile the document several times to get a multi-page table to line up properly
\begin{longtable}{|l|l|}
	\hline
	\textbf{Nome caso d'uso} & UC3: Iscrive a un corso \\ \hline
	\endfirsthead
	%
	\endhead
	%
	\textbf{Portata} & Piattaforma uCOM \\ \hline
	\textbf{Livello} & Obiettivo utente \\ \hline
	\textbf{Attore primario} & Studente \\ \hline
	\textbf{\begin{tabular}[c]{@{}l@{}}Parti interessate \\ e Interessi\end{tabular}} & \begin{tabular}[c]{@{}l@{}}Studente: vuole iscriversi a un corso del Campus.\\ \\ Direzione Campus: vuole che i propri studenti possano\\ iscriversi ai corsi tramite la piattaforma fornita\end{tabular} \\ \hline
	\textbf{Pre-condizioni} & Lo Studente possiede un account sulla piattaforma. \\ \hline
	\textbf{Garanzia di successo} & Lo Studente ha ricevuto conferma dell'operazione. \\ \hline
	\textbf{\begin{tabular}[c]{@{}l@{}}Scenario principale \\ di successo\end{tabular}} & \begin{tabular}[c]{@{}l@{}}1. Lo Studente effettua l'accesso\\ 2. Lo Studente avvia l'operazione di iscrizione a un corso.\\ 3. Lo Studente indica il corso cui vuole iscriversi.\\ 4. Lo Studente invia l'iscrizione al corso.\\ 5. Il Sistema elabora l'iscrizione.\\ 6. Il Sistema conferma la riuscita dell'operazione.\end{tabular} \\ \hline
	\textbf{Estensioni} & \begin{tabular}[c]{@{}l@{}}*a. In qualsiasi momento.Il Sistema non è in grado di funzionare \\ correttamente in un dato momento.\\ \\     1) Il Sistema segnala l'impossibilità di eseguire l'azione.\\ \\     - Lo Studente riprova a eseguire l'azione dopo un certo periodo\\     di tempo.\\ \\ *b. In qualsiasi momento. Il Sistema entra in uno stato di errore\\ irrisolvibile.\\ \\     1) Il Sistema termina la sessione, perdendo i dati.\\ \\     - Lo Studente deve ricominciare l'operazione.\\ \\ *c. In qualsiasi momento. Lo Studente interrompe l'operazione.\\ \\ 1) Il Sistema termina l'operazione.\\ \\ 3a. Lo Studente indica un'opzione non valida o disponibile.\\ \\     1) Il Sistema richiede nuova immissione dei dati allo Studente.\\ \\ 5a. Lo studente non può iscriversi al corso selezionato.\\ \\   1a) Il Sistema indica il motivo per cui non è possibile iscriversi\\ e termina l'operazione.\end{tabular} \\ \hline
	\textbf{Requisiti speciali} & \begin{tabular}[c]{@{}l@{}}- Lo Studente deve inserire dati che siano conformi ai corsi\\ offerti dal Campus\end{tabular} \\ \hline
	\textbf{\begin{tabular}[c]{@{}l@{}}Elenco delle varianti \\ tecnologiche e dei dati\end{tabular}} & \begin{tabular}[c]{@{}l@{}}3) L'inserimento delle informazioni può avvenire attraverso\\ metodi input diversi, come tastiera e mouse o un touchscreen.\\ \\ 3) I Corsi disponibili sono quelli del Registro Corsi di uCOM.\end{tabular} \\ \hline
	\textbf{Frequenza di ripetizione} & bassa \\ \hline
	\textbf{Varie} & \begin{tabular}[c]{@{}l@{}}Si potrebbe prevedere un sistema che permetta l'inserimento\\ offline e l'elaborazione non appena il servizio ritorna disponibile.\\ \\ Si possono prevedere meccanismi di recupero dell'istanza in\\ caso di errori gravi.\end{tabular} \\ \hline
\end{longtable}