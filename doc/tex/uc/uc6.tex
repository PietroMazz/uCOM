% Please add the following required packages to your document preamble:
% \usepackage[normalem]{ulem}
% \useunder{\uline}{\ul}{}
% \usepackage{longtable}
% Note: It may be necessary to compile the document several times to get a multi-page table to line up properly
\begin{longtable}{|l|l|}
	\hline
	\textbf{Nome caso d'uso} & UC6: Gestisce iscrizione corso \\ \hline
	\endfirsthead
	%
	\endhead
	%
	\textbf{Portata} & Piattaforma uCOM \\ \hline
	\textbf{Livello} & Obiettivo utente (CRUD) \\ \hline
	\textbf{Attore primario} & Amministratore \\ \hline
	\textbf{\begin{tabular}[c]{@{}l@{}}Parti interessate \\ e Interessi\end{tabular}} & \begin{tabular}[c]{@{}l@{}}Amministratore: vuole poter gestire la creazione di\\ un'iscrizione a un corso svolto internamente al Campus\\ \\ Direzione Campus: vuole che i corsi che si\\ svolgono all'interno del Campus siano gestite internamente dal\\ personale del Campus (Amministrazione)\end{tabular} \\ \hline
	\textbf{Pre-condizioni} & L'Amministratore possiede un account sulla piattaforma \\ \hline
	\textbf{Garanzia di successo} & \begin{tabular}[c]{@{}l@{}}Una nuova iscrizione è stata creata.\\ L'Amministratore ha ricevuto conferma dell'operazione.\end{tabular} \\ \hline
	\textbf{\begin{tabular}[c]{@{}l@{}}Scenario principale \\ di successo\end{tabular}} & \begin{tabular}[c]{@{}l@{}}1. L'Amministratore effettua l'accesso.\\ 2. L'Amministratore avvia l'iscrizione a un corso.\\ 3. L'Amministratore inserisce nome del corso e \\identificativo dello studente.\\ 4. L'Amministratore aggiunge l'iscrizione al Sistema.\\ 5. Il Sistema aggiunge l'iscrizione al proprio Registro Iscrizioni.\\ 6. Il Sistema conferma la riuscita dell'operazione.\end{tabular} \\ \hline
	\textbf{Estensioni} & \begin{tabular}[c]{@{}l@{}}*a. In qualsiasi momento.Il Sistema non è in grado di funzionare\\ correttamente in un dato momento.\\ \\ \quad1) Il Sistema segnala l’impossibilità a di eseguire l’azione. \\ \\ \quad- L'Amministratore riprova a eseguire l’azione dopo un certo \\ periodo di tempo.\\ \\ *b. In qualsiasi momento. Il Sistema entra in uno stato di errore \\ irrisolvibile. \\ \\ \quad1) Il Sistema termina la sessione, perdendo i dati. \\ \\ \quad- L'Amministratore deve ricominciare l’operazione.\\ \\ *c. In qualsiasi momento. L'Amministratore interrompe \\ l’operazione. \\ \\ \quad1) Il Sistema termina l’operazione.\\ \\ 2a. L'Amministratore avvia la lettura degli iscritti a un corso.\\ \\ \quad1) L'Amministratore inserisce il nome del Corso.\\ \\ \quad2) L'Amministratore richiede la lettura degli iscritti al corso.\\ \\ \quad3) Il Sistema preleva i dati relativi al corso richiesto \\ dal Registro Iscrizioni.\\ \\ \quad4) Il Sistema restituisce gli iscritti al corso richiesto.\end{tabular} \\ \hline
	\textbf{Estensioni} & \begin{tabular}[c]{@{}l@{}}2b. L'Amministratore avvia la modifica di un'iscrizione.\\ \\ \quad1) L'Amministratore inserisce le informazioni\\ relative a un'iscrizione da modificare.\\ \\ \quad2) L'Amministratore modifica l'iscrizione.\\ \\ \quad3) Il Sistema aggiorna l'iscrizione sul Registro Iscrizioni.\\ \\ \quad4) Il Sistema conferma la riuscita dell'operazione.\\ \\ 2c. L'Amministratore avvia l'eliminazione di un'iscrizione.\\ \\ \quad1) L'Amministratore inserisce le informazioni dell'iscrizione\\ da eliminare.\\ \\ \quad2) L'Amministratore elimina l'iscrizione.\\ \\ \quad3) Il Sistema elimina l'iscrizione dal Registro Iscrizioni.\\ \\ \quad4) Il Sistema conferma la riuscita dell'operazione.\\ \\ 3a. L'Amministratore inserisce informazioni non valide. \\ \\ \quad1) Il Sistema richiede nuova immissione dei dati \\ all'Amministratore\end{tabular} \\ \hline
	\textbf{Requisiti speciali} & Nessuno \\ \hline
	\textbf{\begin{tabular}[c]{@{}l@{}}Elenco delle varianti \\ tecnologiche e dei dati\end{tabular}} & \begin{tabular}[c]{@{}l@{}}3) L'inserimento delle informazioni può avvenire attraverso\\ metodi input diversi, come tastiera e mouse o un touchscreen.\\ \\ Le iscrizioni devono essere relative a\\ corsi esistenti nel Registro Corsi.\end{tabular} \\ \hline
	\textbf{Frequenza di ripetizione} & frequenza bassa \\ \hline
	\textbf{Varie} & \begin{tabular}[c]{@{}l@{}}Si potrebbero aggiungere nuove informazioni da memorizzare sul\\ Registro Iscrizioni.\end{tabular} \\ \hline
\end{longtable}