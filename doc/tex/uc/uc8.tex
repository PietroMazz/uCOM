% Please add the following required packages to your document preamble:
% \usepackage[normalem]{ulem}
% \useunder{\uline}{\ul}{}
% \usepackage{longtable}
% Note: It may be necessary to compile the document several times to get a multi-page table to line up properly
\begin{longtable}{|l|l|}
	\hline
	\textbf{Nome caso d'uso} & UC8: Gestisce utente \\ \hline
	\endfirsthead
	%
	\endhead
	%
	\textbf{Portata} & Piattaforma uCOM \\ \hline
	\textbf{Livello} & Obiettivo utente (CRUD) \\ \hline
	\textbf{Attore primario} & System Admin \\ \hline
	\textbf{\begin{tabular}[c]{@{}l@{}}Parti interessate \\ e Interessi\end{tabular}} & \begin{tabular}[c]{@{}l@{}}SystemAdmin: vuole poter gestire la creazione di un utente\\ per poter accedere alla piattaforma\\ \\ Direzione Campus: necessita che l'accesso di ogni utente\\ sia verificato, per garantire sicurezza al sistema, e che solo un\\ utente autorizzato, quale il SystemAdmin possa modificare\\ gli utenti che possono accedere al Sistema\end{tabular} \\ \hline
	\textbf{Pre-condizioni} & Il SystemAdmin possiede un account sulla piattaforma \\ \hline
	\textbf{Garanzia di successo} & Un nuovo utente può avere accesso alla piattaforma \\ \hline
	\textbf{\begin{tabular}[c]{@{}l@{}}Scenario principale \\ di successo\end{tabular}} & \begin{tabular}[c]{@{}l@{}}1. Il SystemAdmin effettua l'accesso.\\ 2. Il SystemAdmin avvia la creazione di un utente.\\ 3. Il SystemAdmin inserisce username e ruolo per l'utente.\\ 4. Il SystemAdmin aggiunge l'utente al Sistema.\\ 5. Il Sistema aggiunge l'utente al proprio Registro Utenti.\\ 6. Il Sistema conferma la riuscita dell'operazione.\end{tabular} \\ \hline
	\textbf{Estensioni} & \begin{tabular}[c]{@{}l@{}}*a. In qualsiasi momento.Il Sistema non `e in grado di funzionare\\ correttamente in un dato momento.\\ \\ \quad1) Il Sistema segnala l’impossibilità a di eseguire l’azione. \\ \\ \quad- Il SystemAdmin riprova a eseguire l’azione dopo un certo \\ periodo di tempo.\\ \\ *b. In qualsiasi momento. Il Sistema entra in uno stato di errore \\ irrisolvibile. \\ \\ \quad1) Il Sistema termina la sessione, perdendo i dati. \\ \\ \quad- Il SystemAdmin deve ricominciare l’operazione.\\ \\ *c. In qualsiasi momento. Il SystemAdmin \\ interrompe l’operazione. \\ \\ \quad1) Il Sistema termina l’operazione.\\ \\ 2a. Il SystemAdmin avvia la lettura dei dati di un utente.\\ \\ \quad 1) Il SystemAdmin inserisce l'identificativo dell'utente.\\ \\ \quad 2) Il SystemAdmin richiede la lettura dell'utente.\\ \\ \quad 3) Il Sistema preleva i dati dell'utente richiesto \\ dal Registro Utenti.\\ \\ \quad4) Il Sistema restituisce i dati dell'utente richiesto.\end{tabular} \\ \hline
	\textbf{Estensioni} & \begin{tabular}[c]{@{}l@{}}2b. Il SystemAdmin avvia la modifica di un utente.\\ \\ \quad1) Il SystemAdmin inserisce le informazioni dell'utente\\da modificare.\\ \\ \quad2) Il SystemAdmin modifica l'utente.\\ \\ \quad3) Il Sistema aggiorna l'utente sul RegistroUtenti.\\ \\ \quad4) Il Sistema conferma la riuscita dell'operazione.\\ \\ 2c. Il SystemAdmin avvia l'eliminazione di un utente.\\ \\ \quad1) Il SystemAdmin inserisce le informazioni dell'utente\\  da eliminare.\\ \\ \quad2) Il SystemAdmin elimina l'utente.\\ \\ \quad 3) Il Sistema elimina l'utente dal RegistroUtenti.\\ \\ \quad4) Il Sistema conferma la riuscita dell'operazione.\\ \\ 3a. Il SystemAdmin inserisce informazioni non valide. \\ \\ \quad1) Il Sistema richiede nuova immissione dei dati\\ al SystemAdmin\end{tabular} \\ \hline
	\textbf{Requisiti speciali} & Nessuno \\ \hline
	\textbf{\begin{tabular}[c]{@{}l@{}}Elenco delle varianti \\ tecnologiche e dei dati\end{tabular}} & \begin{tabular}[c]{@{}l@{}}3) L'inserimento delle informazioni può avvenire attraverso\\ metodi d'input diversi, come tastiera e mouse o un touchscreen.\end{tabular} \\ \hline
	\textbf{Frequenza di ripetizione} & frequenza medio-bassa \\ \hline
	\textbf{Varie} & \begin{tabular}[c]{@{}l@{}}Si potrebbero aggiungere nuove informazioni da memorizzare sul\\ Registro Utenti.\end{tabular} \\ \hline
\end{longtable}